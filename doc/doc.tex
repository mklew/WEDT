% Created 2013-11-12 Tue 20:00
\documentclass[12pt, a4paper]{article}
\usepackage{polski}
\usepackage[utf8x]{inputenc}
\usepackage[polish]{babel} 
\usepackage{geometry}
\usepackage{hyperref}
\usepackage{amsmath}
\usepackage[numbers]{natbib}
\usepackage{algorithm}
\usepackage{algpseudocode}
\usepackage{float}
\usepackage{graphicx}
\usepackage{listings}

\lstdefinelanguage{scala}{
  morekeywords={abstract,case,catch,class,def,%
    do,else,extends,false,final,finally,%
    for,if,implicit,import,match,mixin,%
    new,null,object,override,package,%
    private,protected,requires,return,sealed,%
    super,this,throw,trait,true,try,%
    type,val,var,while,with,yield},
  otherkeywords={=>,<-,<\%,<:,>:,\#,@},
  sensitive=true,
  morecomment=[l]{//},
  morecomment=[n]{/*}{*/},
  morestring=[b]",
  morestring=[b]',
  morestring=[b]"""
}

\author{Marek Lewandowski, Marcin Król}
\date{}
\title{System skanujący opinie użytkowników na forum pod jakimś artykułem i określający zbiorczą opinię użytkowników \\ 
        WEDT - Dokumentacja wstępna}
\begin{document}

\maketitle

\section{Założenia}
W projekcie przyjeliśmy następujące założenia:

\begin{itemize}
\item komentarze są wyświetlane statycznie, czyli nie potrzebny jest javascript do obejrzenia komentarzy
\item ilość komentarzy powinna wynosić co najmniej $N$, gdzie $N$ określimy empirycznie
\item komentarz powinien posiadać co najmniej $W$, gdzie $W$ określimy empirycznie
\end{itemize}

\section{System}

System można podzielić na dwa moduły:
\begin{itemize}
\item Crawler  - zajmuje się wydobyciem komentarzy ze strony i poszukiwaniem następnych stron
\item Sentiment Analyzer - określa sentyment komentarza
\end{itemize}

\subsection{Crawler}

\subsubsection{Metoda wydobycia komentarzy}
\begin{enumerate}
\item Zaczynamy od wyszukania powtarzających się na stronie elementów spełniających określone kryteria np.: id albo klasa elementu zawiera słowo komentarz. Przyjmujemy, że znalezione elementy to komentarze.
\item Dla każdego elementu przeszukujemy jego strukturę szukając podelementu zawierającego treść z co najmniej $W$ słowami.
\item Wynikiem są znalezione treści komentarzy
\end{enumerate}

\subsubsection{Metoda paginacji}
Paginacja opiera się na metodzie słów kluczowych. Szukać będziemy elementów typu $<a></a>$ które zawierają takie słowa jak: “Dalej”, “Następna” itp.

\subsection{Sentiment Analyzer}

\subsubsection{Metoda analizy sentymentu}

opis metody z wysokiego poziomu:
\begin{enumerate}
\item LanguageTool do korekcji treści
\item Stemizacja
\item Zliczanie stemów pozytywnych i negatywnych na podstawie słownika słów pozytywnych i negatywnych
\item Określenie sentymentu na podstawie stosunku liczby słów pozytywnych i negatywnych
\item Określenie istotności
\end{enumerate}

\subsubsection{LangagueTool}
Komentarze często zawierają błędy typu literówki, zła gramatyka. Użycie takich słów do dalszej analizy może być problematyczne, ponieważ słowniki zawierają słowa zapisane w sposób poprawny. LanguageTool skoryguje treść komentarza i taka poprawiona treść będzie użyta dalej.

\subsubsection{Stemizacja}
Język polski jest językiem fleksyjnym. Słowniki słów pozytywnych i negatywnych nie zawierają wszystkich słów we wszystkich formach. W związku z tym poprzez używanie jedynie stemów mamy większą szansę, na zakwalifikowanie danego słowa do pozytywnych lub negatywnych.

\subsubsection{Zliczanie stemów}

Korzystając z dwóch wcześniej przygotowanych słowników: słownika słów pozytywnych oraz negatywnych sprawdzamy czy dane słowo należy do jednego lub drugiego słownika. Jeśli należy to jest kwalifikowane jako pozytywne lub negatywne. W przeciwnym przypadku pozostaje jako neutralne. 

Jako, że korzystamy ze stemów słów słowniki również zostaną poddane stemizacji.

\subsubsection{Określenie sentymentu}

Posiadając słowa określone jako pozytywne, negatywne i neutralne możemy określić zabarwienie sentymentu danego komentarza. Skorzystamy tutaj z formuły, którą określimy empirycznie. Wejściem formuły będą 3 zmienne, a wyjściem wartość z zakresu $(-1;1)$ określająca zabarwienie sentymentu.

\subsubsection{Określenie istotności}

Istotność określimy na podstawie ilości słów w komentarzu. Dłuższy komentarz uznamy za bardziej istotny.


% \nocite{*}
% \bibliographystyle{plainnat}
% \bibliography{bibliography}
\end{document}
